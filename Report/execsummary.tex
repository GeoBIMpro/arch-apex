
Part I: Understanding algorithms and models for AVs
\begin{itemize}
	\item Creation of realistic planning stack
	\item Creation of simulation environment for planning stack
	\item Validation of planning stack on real vehicle
\end{itemize}

Part II: Verifying scenarios involving lane changes on periodic road structures
\begin{itemize}
	\item Formalization of vehicle model and environment as a hybrid system
	\item Interface between planning stack and verification instances
	\item Proof of chaining for verified decisions
	\item Infinite time property for periodic roads
	\item Verification of a lane change scenario
\end{itemize}

Part III: Alternate formal representations of scenarios via geometric methods, computation modes, and differential inclusions
\begin{itemize}
	\item Implementation and simulation of a new planning mechanism: pure pursuit
	\item Addition of differential inclusions to dReach language
	\item New hybrid systems representation of AV which does not require chaining
	\item New hybrid systems representation of traffic participant behaviors
	\item Ongoing work to understand scalability
\end{itemize}

Part IV: Using our tools, lessons from implementations and installation
\begin{itemize}
	\item Understanding the limitations of dReach environment in the design process
	\item New recommended workflow for development of AV algorithms
\end{itemize}

Part V: Conclusions and Future Work:
\begin{itemize}
	\item Must build simulation tools
	\item Likely will need to find clever ways to identify portions of scenarios such that verification is scalable
	\item Finish building bottom up components such that users can work in top down manner to specify scenarios
\end{itemize}

Key Results:\\
\begin{itemize}
	\item Build and Test Planning Stack
	\item Formalize planning stack and verify lane change scenario
	\item Understand alternative stack representations which relax requirements for verification
	\item Tool available for download
\end{itemize}
